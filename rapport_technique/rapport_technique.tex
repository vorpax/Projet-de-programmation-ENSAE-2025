\documentclass[11pt, a4paper]{article}

% Paquets essentiels
\usepackage[utf8]{inputenc}
\usepackage[T1]{fontenc}
\usepackage{amsmath, amssymb, amsthm}
\usepackage{graphicx}
\usepackage{geometry}
\usepackage{hyperref}
\usepackage{xcolor}
\usepackage{listings}
\usepackage{enumitem}
\usepackage{booktabs}
\usepackage{multicol}
\usepackage{algorithm}
\usepackage{algpseudocode}
\usepackage[normalem]{ulem}

% Configuration de la géométrie de la page
\geometry{a4paper, margin=2.5cm}

% Configuration de l'hyperlien
\hypersetup{
    colorlinks=true,
    linkcolor=blue,
    filecolor=magenta,
    urlcolor=cyan,
    citecolor=red,
}

% Configuration de l'environnement de code pour UTF-8
\lstset{
    basicstyle=\ttfamily\footnotesize,
    numbers=left,
    numberstyle=\tiny,
    frame=single,
    breaklines=true,
    captionpos=b,
    showstringspaces=false,
    tabsize=2,
    language=Python,
    literate={é}{{\'e}}1 {à}{{\`a}}1 {è}{{\`e}}1 {ê}{{\^e}}1 {û}{{\^u}}1 {ô}{{\^o}}1
             {ç}{{\c{c}}}1 {œ}{{\oe}}1 {ù}{{\`u}}1 {î}{{\^{\i}}}1
             {É}{{\'E}}1 {À}{{\`A}}1 {È}{{\`E}}1 {Ê}{{\^E}}1 {Û}{{\^U}}1 {Ô}{{\^O}}1
             {Ç}{{\c{C}}}1 {Œ}{{\OE}}1 {Ù}{{\`U}}1 {Î}{{\^I}}1
}

% Personnalisation des titres de section
\usepackage{sectsty}
\sectionfont{\Large\bfseries\color{blue}}
\subsectionfont{\large\bfseries\color{blue}}
\subsubsectionfont{\normalsize\bfseries\color{blue}}

% Informations du document
\title{\textbf{Rapport de Projet : Matching sur Grille}}
\author{
    Alexandre Houard \& Octave Hedarchet \\
    ENSAE Paris
}
\date{\today}

% Définition d'environnements théorèmes
\newtheorem{theorem}{Théorème}[section]
\newtheorem{definition}{Définition}[section]
\newtheorem{proposition}{Proposition}[section]
\newtheorem{corollary}{Corollaire}[section]
\newtheorem{lemma}{Lemme}[section]
\newtheorem{example}{Exemple}[section]

\begin{document}

\maketitle
\thispagestyle{empty} % Enlève la numérotation de la page de titre

\newpage
\tableofcontents
\newpage

\section{Introduction}
\label{sec:introduction}

Le problème du matching dans les graphes est un problème fondamental en informatique et en théorie des graphes. Il consiste à trouver un ensemble de paires d'éléments qui satisfont certaines contraintes, généralement dans le but de minimiser ou maximiser un critère donné, comme le coût. Ce problème est transversal à divers domaines, notamment dans les réseaux de transport, dans l'appariement de données dans les systèmes de recommandation ou l'optimisation de processus industriels.

Dans ce contexte, l'objectif principal de ce rapport est de traiter de la résolution du problème de correspondance dans un cadre bidimensionnel, représenté sous forme de grille, où les cellules doivent être appariées selon certaines règles de coût et de validité. Nous explorons plusieurs algorithmes de résolution qui permettent de traiter ce problème de manière optimale, en minimisant le coût des appariements tout en respectant des contraintes spécifiques, telles que l'adjacence des cellules et l'interdiction de certaines paires en fonction de leurs caractéristiques.

Ce rapport présente une analyse approfondie des défis techniques, des choix algorithmiques et des structures de données utilisés dans l'implémentation de notre système de matching sur grille. Le projet comprend plusieurs algorithmes de résolution (Greedy, Ford-Fulkerson, Hongrois) appliqués à un problème d'appariement de cellules sur une grille colorée.

\subsection{Contexte et Motivation}

L'objectif de ce projet de programmation était de résoudre un problème de matching biparti sur une grille colorée, où des cellules doivent être appariées selon des règles de compatibilité de couleurs et de positions. Ce type de problème a des applications dans divers domaines comme l'allocation de ressources, la planification et l'optimisation combinatoire.

Le rapport détaille d'abord des choix algorithmiques utilisés (méthode de Ford Fulkerson, algorithme hongrois, etc.) pour résoudre ce problème. Ensuite, il présente une évaluation de la performance des algorithmes en utilisant l'argument de leurs complexité. Puis, les résultats de ce projet sont présentés avant de proposer une extension sous forme de jeu interactif.

\subsection{Objectifs}

Les objectifs principaux de ce projet étaient:
\begin{itemize}
    \item Modéliser une grille colorée comme un problème de matching biparti
    \item Implémenter différents algorithmes de résolution (Greedy, Ford-Fulkerson, Hongrois)
    \item Comparer les performances et la qualité des solutions obtenues
    \item Fournir une interface visuelle pour observer les matchings générés
    \item Développer une extension interactive permettant de jouer à ce jeu
\end{itemize}

\subsection{Aperçu du Rapport}

Ce rapport ne se veut aucunement exhaustif quant au code du projet, son fonctionnement et son implémentation. L'objectif est avant tout de fournir un aperçu des éléments saillants au fil du projet, en mettant l'accent sur les défis techniques rencontrés et les solutions apportées.

\subsection{Outils et Méthodologie}

\subsubsection{Git}

Afin de collaborer au mieux ensemble, nous avons utilisé git tout au long du projet. Même si nous n'étions pas habitués à l'utiliser pour travailler en équipe mais surtout lors de projets individuels pour profiter du versionnage, ce dernier nous a été utile.

Afin de suivre les meilleures pratiques en terme de versionnage de code, nous travaillions avec 3 branches : la branche main, la branche Alexandre et la branche Octave. Les modifications n'étaient jamais faites sur le main mais toujours sur les branches respectives avant de merger les changements avec une pull-request.
Nous aurions d'ailleurs apprécié qu'une courte introduction à git nous soit faite au vu de l'importance de cette technologie.

\subsubsection{IDE et Linting}

Comme indiqué, nous utilisions tous les deux Visual Studio Code.
Afin de garantir un code propre, agréable à lire et à comprendre, nous avons utilisé successivement Pylint puis Ruff. Ce choix de changement de formater et de Linter a été avant tout motivé par la vitesse de Ruff qui nous donnait des conseils et avertissements au fur et à mesure de l'écriture du code là où pylint avait un peu plus de latence, ce qui pouvait être gênant dans l'itération successive du code.

\section{Modélisation du problème}
\label{sec:modelisation}

\subsection{Grille et graphe bipartite}

La modélisation du problème comme un problème de matching biparti a constitué le premier défi conceptuel. Il a fallu transformer une grille 2D en un graphe biparti où:
\begin{itemize}
    \item Les cellules de parité paire (i+j est pair) forment un ensemble de sommets
    \item Les cellules de parité impaire forment l'autre ensemble
    \item Les arêtes représentent les paires valides selon des règles de compatibilité de couleurs
\end{itemize}

La matrice \texttt{MatriceCouleurOk} définit ces règles de compatibilité entre couleurs:
\begin{lstlisting}[caption=Matrice de compatibilite des couleurs]
BlancCombinaisonOk = [1, 1, 1, 1, 0]
RougeCombinaisonOk = [1, 1, 1, 0, 0]
BleuCombinaisonOk = [1, 1, 1, 0, 0]
VertCombinaisonOk = [1, 0, 0, 1, 0]
NoirCombinaisonOk = [0, 0, 0, 0, 0]
\end{lstlisting}

Une paire entre deux cellules n'est valide que si l'entrée correspondante dans cette matrice est 1, ce qui représente une règle métier complexe à intégrer dans les algorithmes de matching.

\section{Choix algorithmiques}
\label{sec:algorithmes}

Dans ce projet, plusieurs algorithmes fondamentaux ont été utilisés pour résoudre le problème. Ces algorithmes incluent principalement l'algorithme Hongrois, l'algorithme de Ford-Fulkerson et une approche gloutonne. Ces choix ont été motivés par la nature du problème de correspondance et les besoins d'optimisation associés. Dans cette section, nous allons détailler les caractéristiques de ces algorithmes, leur fonctionnement, et pourquoi ils ont été sélectionnés.

\subsection{Le greedy solver}

Le greedy solver n'a pas posé de difficultés particulières dans son implémentation. En revanche, nous avons été très surpris de sa complexité algorithmique, de l'ordre de O(p²) où p est le nombre de paires valides.

\begin{lstlisting}[caption=Implementation du solveur glouton]
def run(self) -> list[list[tuple[int, int]]]:
    # Tri des paires: O(p log p)
    all_pairs_sorted = self.grid.all_pairs().copy()
    all_pairs_sorted.sort(key=self.grid.cost)
    
    # Pour chaque paire, on filtre les paires incompatibles: O(p^2)
    while len(all_pairs_sorted) > 0:
        filtered_list = []
        for pair in all_pairs_sorted:
            if pair[0] not in chosen_cells and pair[1] not in chosen_cells:
                filtered_list.append(pair)
        # ...
\end{lstlisting}

Elle s'explique aisément par la méthode employée et est inhérente à l'implémentation d'un \textit{greedy algorithm}. À chaque étape, il nous a fallu recalculer les paires possibles, ce qui était très coûteux.

Nous avions envisagé le fait de potentiellement retirer des éléments de la liste mais cela aurait été équivalent à recréer une liste à chacune des étapes, chose pire du point de vue du nombre d'opérations que de simplement rajouter des éléments à la liste "paires possibles" comme nous l'avons fait.

Nous avons décidé de ne pas plus explorer une telle approche dans la mesure où l'algorithme greedy présente des limites qui lui sont propres : une succession d'optima locaux (paire la moins coûteuse à chaque étape) ne garantit pas un optimum global à la fin (le matching n'est pas nécessairement optimal).

\subsection{Ford-Fulkerson}

L'Algorithme de Ford-Fulkerson fut un premier pas dans l'implémentation d'algorithmes de matching. Il propose une méthode efficace afin de résoudre le problème mais sa portée est assez limitée dans la mesure où il ne s'applique concrètement que dans le cas où la valeur de toutes les cellules est égale à 1.

\subsubsection{Fonctionnement de l'algorithme de Ford-Fulkerson}

L'algorithme repose sur la modélisation du problème de correspondance comme un problème de flux dans un réseau. Le graphe est divisé en deux sous-ensembles : un ensemble de cellules de "source" (les cellules paires) et un ensemble de "puits" (les cellules restantes). L'algorithme cherche à maximiser le flux entre ces deux ensembles, ce qui correspond à maximiser le nombre de paires valides formées dans le problème de correspondance.

Voici comment il fonctionne :

\begin{itemize}
    \item Création du graphe de résidu : Chaque paire de cellules possibles est considérée comme un arc entre deux nœuds du graphe. Le flux maximal est déterminé en augmentant progressivement les flux possibles entre les nœuds du graphe.
    \item Recherche de chemins augmentants : À chaque itération, on recherche un chemin augmentant du nœud source au nœud puits, c'est-à-dire un chemin où il est possible d'augmenter le flux.
    \item Augmentation du flux : Une fois un chemin trouvé, le flux sur ce chemin est augmenté, ce qui augmente le nombre total de correspondances.
\end{itemize}

\begin{lstlisting}[caption=Implementation de Ford-Fulkerson]
def ford_fulkerson(self) -> int:
    max_flow = 0
    
    # Recherche de chemins augmentants: O(V*E) dans le pire cas
    path = self.find_augmenting_path()
    while path:
        # Mise a jour du graphe residuel: O(E)
        # ...
        max_flow += min_capacity
        path = self.find_augmenting_path()
\end{lstlisting}

Malheureusement, il n'est pas évident de le généraliser dans la mesure où le fait que la valeur de toutes les cellules soit égale à 1 garantissait le respect de la contrainte "une cellule ne peut faire partie que d'au plus une paire".

Dans le cas général, nous avons décidé de ne pas approfondir une généralisation de l'implémentation du SolverFulkerson pour se concentrer sur l'algorithme Hongrois.

\subsection{Algorithme Hongrois}

L'implémentation de l'algorithme hongrois a représenté le défi technique le plus significatif:
\begin{itemize}
    \item Sa mise en œuvre nécessite une compréhension profonde de l'algèbre linéaire
    \item Les chemins augmentants et les potentiels duaux sont conceptuellement difficiles
    \item L'optimisation pour maintenir une complexité O(n³) demande une attention particulière
\end{itemize}

\begin{lstlisting}[caption=Coeur de l'algorithme hongrois]
def _find_augmenting_path(self, cost: np.ndarray, u: np.ndarray, v: np.ndarray, 
                       path: np.ndarray, row4col: np.ndarray, current_row: int) -> tuple:
    min_value = 0
    num_remaining = cost.shape[1]
    remaining = np.arange(cost.shape[1])[::-1]
    
    SR = np.full(cost.shape[0], False, dtype=bool)
    SC = np.full(cost.shape[1], False, dtype=bool)
    
    shortest_path_costs = np.full(cost.shape[1], np.inf)
    sink = -1
    
    # Boucle de recherche du chemin augmentant
    while self._find_short_augpath_while_cond(...):
        # Complexe implementation detaillee...
\end{lstlisting}

Notre implémentation est un mélange de l'algorithme Hongrois classique et de l'approche Primal-Dual, permettant d'obtenir des solutions optimales pour tous les cas de test.

\section{Analyse de la complexité algorithmique}
\label{sec:complexite}

\subsection{Algorithme Greedy (SolverGreedy)}

\textbf{Complexité temporelle}: O(p²) où p est le nombre de paires valides

L'algorithme glouton présente l'avantage d'être simple et rapide pour des grilles de petite taille, mais il peut produire des solutions sous-optimales, comme le montrent les résultats des benchmarks où il n'atteint des solutions optimales que dans 50\% des cas sur les petites grilles et 11,1\% sur les grilles moyennes.

\subsection{Algorithme de Ford-Fulkerson (SolverFulkerson)}

\textbf{Complexité temporelle}: O(V·E²) où V est le nombre de sommets et E le nombre d'arêtes

L'algorithme construit un graphe résiduel où:
\begin{itemize}
    \item Une source est connectée à toutes les cellules de parité paire
    \item Toutes les cellules de parité impaire sont connectées à un puits
    \item Les cellules de parité paire sont connectées aux cellules adjacentes de parité impaire
\end{itemize}

Cette représentation permet d'appliquer l'algorithme de flux maximum pour résoudre le problème de matching biparti. La fonction \texttt{find\_augmenting\_path} avec une complexité de O(V+E) est appelée potentiellement O(V·E) fois, ce qui donne une complexité globale de O(V·E²).

\subsection{Algorithme Hongrois (SolverHungarian)}

\textbf{Complexité temporelle}: O(n³) où n est la dimension de la matrice de coût

\begin{lstlisting}[caption=Appel principal de l'algorithme hongrois]
def linear_sum_assignment(self, cost: np.ndarray, maximize: bool = False) -> tuple[np.ndarray, np.ndarray]:
    # Implementation de l'algorithme hongrois
    # ...
    # Recherche de chemins augmentants avec mise a jour des potentiels
    for current_row in range(cost.shape[0]):
        cost, u, v, path, row4col, col4row = self._lsa_body(
            cost, u, v, path, row4col, col4row, current_row
        )
\end{lstlisting}

L'algorithme hongrois est le plus efficace pour trouver des solutions optimales, comme le montrent les benchmarks où il atteint 100\% de solutions optimales sur toutes les tailles de grilles. Sa complexité cubique le rend toutefois plus coûteux pour les très grandes grilles.

\section{Structures de données}
\label{sec:structures}

\subsection{Représentation de la grille}

La classe \texttt{Grid} utilise plusieurs structures complémentaires:

\begin{lstlisting}[caption=Initialisation de la grille]
def __init__(self, n: int, m: int, color: list[list[int]] = None, value: list[list[int]] = None) -> None:
    self.n = n
    self.m = m
    self.color = color if color else [[0 for j in range(m)] for i in range(n)]
    self.value = value if value else [[1 for j in range(m)] for i in range(n)]
    self.colors_list: list[str] = ["w", "r", "b", "g", "k"]
    self.cells_list: list[Cell] = []
    self.cells: list[list[Cell]] = []
\end{lstlisting}

Cette double représentation (tableaux 2D et objets \texttt{Cell}) permet:
\begin{itemize}
    \item Un accès direct par coordonnées via \texttt{color[i][j]} et \texttt{value[i][j]}
    \item Une manipulation orientée-objet via \texttt{cells[i][j]}
    \item Un accès séquentiel via \texttt{cells\_list}
\end{itemize}

Les paires sont représentées comme des tuples de tuples \texttt{((i1,j1), (i2,j2))}, ce qui facilite l'accès direct aux informations de cellules tout en maintenant leur relation.

\subsection{Représentation du graphe pour le matching}

La classe \texttt{SolverFulkerson} utilise un dictionnaire pour représenter le graphe d'adjacence:

\begin{lstlisting}[caption=Initialisation du graphe d'adjacence]
def adjacency_graph_init(self) -> None:
    self.residual_graph["source"] = {}
    self.residual_graph["sink"] = {}
    
    # Construction du graphe...
    for i in range(self.grid.n):
        for j in range(self.grid.m):
            cell_id = f"cell_{i}_{j}"
            self.residual_graph[cell_id] = {}
\end{lstlisting}

Cette structure offre une flexibilité pour les mises à jour dynamiques du graphe résiduel pendant l'exécution de l'algorithme de Ford-Fulkerson.

La classe \texttt{SolverHungarian} utilise des tableaux numpy pour la matrice de coût:

\begin{lstlisting}[caption=Construction de la matrice de cout]
# Construction de la matrice de cout
cost_matrix = np.zeros((max_dim, max_dim))
for u, v in pairs:
    # Calcul des poids...
    cost_matrix[even_to_idx[u], odd_to_idx[v]] = weight
\end{lstlisting}

Ces tableaux numpy permettent des opérations matricielles optimisées essentielles pour l'algorithme hongrois.

\section{Résultats Expérimentaux}
\label{sec:resultats}

\subsection{Protocole Expérimental}

Nous avons testé nos trois algorithmes sur différentes tailles de grilles:
\begin{itemize}
    \item Petites grilles: grid0x.in (environ 6 cellules)
    \item Grilles moyennes: grid1x.in (environ 20-30 cellules)
    \item Grandes grilles: grid2x.in (environ 100 cellules)
\end{itemize}

Pour chaque algorithme et chaque grille, nous avons mesuré:
\begin{itemize}
    \item Le temps d'exécution
    \item Le score obtenu
    \item La déviation par rapport au score optimal
    \item Le nombre de paires formées
\end{itemize}

\subsection{Présentation des Résultats}

\subsubsection{Résultats sur petites grilles}

\begin{lstlisting}[caption=Extrait des resultats sur petites grilles]
================================================================================
Small Grid Benchmark Results
================================================================================
solver,grid,grid_size,time,score,best_score,deviation,pairs,quality
SolverGreedy,grid00.in,2x3,0.000s,14,12,16.67%,3,Suboptimal
SolverFulkerson,grid00.in,2x3,0.000s,14,12,16.67%,3,Suboptimal
SolverHungarian,grid00.in,2x3,0.000s,12,12,0.00%,3,Optimal
\end{lstlisting}

Statistiques des solveurs:
\begin{itemize}
    \item SolverFulkerson: 66.7\% de solutions optimales
    \item SolverGreedy: 50.0\% de solutions optimales
    \item SolverHungarian: 100.0\% de solutions optimales
\end{itemize}

\subsubsection{Résultats sur grilles moyennes}

Statistiques des solveurs:
\begin{itemize}
    \item SolverFulkerson: 66.7\% de solutions optimales
    \item SolverGreedy: 11.1\% de solutions optimales
    \item SolverHungarian: 100.0\% de solutions optimales
\end{itemize}

\subsubsection{Résultats sur grandes grilles}

Pour les grandes grilles (grid2x.in), seul l'algorithme hongrois a été testé en raison de sa supériorité. Par exemple, sur grid21.in:

\begin{lstlisting}[caption=Resultat sur une grande grille]
Completed SolverHungarian on grid21.in: Score=1686, Best=1686, Time=93.646s
\end{lstlisting}

\subsection{Analyse des résultats}

Ces résultats montrent que:
\begin{enumerate}
    \item \textbf{L'algorithme hongrois} est systématiquement le plus précis, trouvant des solutions optimales dans 100\% des cas testés.
    \item \textbf{L'algorithme de Ford-Fulkerson} offre un bon compromis, avec 66,7\% de solutions optimales.
    \item \textbf{L'algorithme glouton} est généralement le plus rapide mais produit souvent des solutions sous-optimales, particulièrement sur les grilles moyennes où son taux de réussite chute à 11,1\%.
\end{enumerate}

Le choix de l'algorithme dépend donc du contexte d'utilisation:
\begin{itemize}
    \item Pour des solutions optimales: l'algorithme hongrois
    \item Pour un bon compromis performance/qualité: l'algorithme de Ford-Fulkerson
    \item Pour des approximations rapides: l'algorithme glouton
\end{itemize}

\section{Extension : jeu interactif}
\label{sec:jeu}

Nous avons rajouté au projet un jeu interactif où les joueurs sélectionnent des cellules adjacentes pour former des paires, dans le but de minimiser la masse totale de la solution de correspondance. L'interface graphique est réalisée avec la bibliothèque PyGame, et plusieurs algorithmes de résolution sont implémentés pour offrir différentes options de jeu, notamment des jeux solo, "contre un autre joueur" et "contre une IA".

Le jeu utilise un tableau à double entrée représentant la grille de cellules où chaque cellule a une couleur et une valeur associée. Les joueurs interagissent avec cette grille, en sélectionnant des cellules adjacentes pour former des paires valides et minimiser le coût global.

\subsection{Modes de jeu}

Le jeu propose plusieurs modes de jeu :

\begin{itemize}
    \item \textbf{Mode solo} : Le joueur joue seul, essayant de minimiser le coût de ses paires.
    \item \textbf{Mode 2 joueurs} : Deux joueurs s'affrontent pour former des paires avec les cellules adjacentes.
    \item \textbf{Mode contre un ordinateur} : Le joueur affronte une intelligence artificielle, avec la possibilité de choisir entre différents algorithmes (Greedy, Fulkerson, ou Hongrois).
\end{itemize}

Cette extension ludique permet non seulement de démontrer l'application pratique des algorithmes développés, mais offre également un moyen interactif de comprendre le problème du matching sur grille.

A votre tour de jouer !

\section{Bonnes pratiques de développement}
\label{sec:bonnes_pratiques}

\subsection{Organisation du code}

Le code est organisé selon le principe de séparation des préoccupations:
\begin{itemize}
    \item La classe \texttt{Grid} gère la représentation des données
    \item La hiérarchie de classes \texttt{Solver} fournit différentes implémentations d'algorithmes
    \item Le module principal orchestre le flux de travail
\end{itemize}

\begin{lstlisting}[caption=Hierarchie des solveurs]
class Solver:
    """Base class with common functionality"""

class SolverEmpty(Solver):
    """Empty implementation for testing"""

class SolverGreedy(Solver):
    """Greedy implementation"""

class SolverFulkerson(Solver):
    """Ford-Fulkerson implementation"""

class SolverHungarian(Solver):
    """Hungarian algorithm implementation"""
\end{lstlisting}

\subsection{Documentation et gestion des erreurs}

Le code inclut une documentation complète avec docstrings et annotations de complexité:

\begin{lstlisting}[caption=Exemple de documentation]
def is_pair_forbidden(self, pair: list[tuple[int, int]]) -> bool:
    """
    Returns True if the pair is forbidden and False otherwise.
    A bit more complex and relevant than simply checking if one of the cells is black.

    Parameters:
    -----------
    pair: list[tuple[int, int]]
        A pair of cells represented as a list of two tuples [(i1, j1), (i2, j2)]
        where (i1, j1) are the coordinates of the first cell and
        (i2, j2) are the coordinates of the second cell

    Returns:
    --------
    bool
        True if the pair is forbidden, False otherwise

    Raises:
    -------
    IndexError
        If either cell's coordinates are out of bounds
        
    Time Complexity: O(1)
        Constant time as it only involves direct index lookup and matrix access.
    """
\end{lstlisting}

La gestion des erreurs est implémentée à travers des vérifications de limites et des validations d'entrée:

\begin{lstlisting}[caption=Exemple de gestion d'erreur]
if i < 0 or i >= self.n or j < 0 or j >= self.m:
    raise IndexError("Cell coordinates out of bounds")
\end{lstlisting}

\subsection{Tests et benchmarks}

Le projet inclut une suite de tests complète:

\begin{lstlisting}[caption=Exemple de test de benchmark]
def test_benchmark_small_grids(self):
    """
    Benchmark all solvers on small grid files and generate a performance report.
    """
    # Get small grids (grid0x.in)
    small_grids = [name for name in self.best_scores.keys() 
                  if re.match(r'grid0[0-5]\.in', name)]
\end{lstlisting}

Ces tests vérifient non seulement la fonctionnalité mais aussi la performance, permettant une comparaison systématique des différents algorithmes.

\section{Fonctionnalités visuelles}
\label{sec:visualisation}

\subsection{La fonction plot}

Afin d'afficher la grille de manière similaire à ce qui avait été fait dans le descriptif du projet, nous avons opté pour la librairie matplotlib.
Nous étions initialement partis pour un affichage dans le terminal (TUI / CLI).

Il nous a fallu utiliser plusieurs astuces afin d'afficher d'une part le cadrillage de la grille et de l'autre les chiffres en abcisse et en ordonnée placés au niveau des ticks qu'il nous a fallu décaler.

\section{Difficultés rencontrées}
\label{sec:difficultes}

\subsection{Représentation duale des données}

La représentation duale des données (tableau 2D et objets Cell) a nécessité une synchronisation constante:

\begin{lstlisting}[caption=Initialisation des cellules]
def cell_init(self) -> None:
    for i in range(self.n):
        self.cells.append([])
        for j in range(self.m):
            self.cells[i].append(Cell(i, j, self.color[i][j], self.value[i][j]))

    self.cells_list = [
        Cell(i, j, self.color[i][j], self.value[i][j])
        for i in range(self.n)
        for j in range(self.m)
    ]
\end{lstlisting}

Cette dualité offre de la flexibilité mais introduit de la redondance dans le code et augmente le risque d'incohérences.

\subsection{Implémentation de l'algorithme hongrois}

La fonction \texttt{\_find\_augmenting\_path} dans la classe \texttt{SolverHungarian} a été particulièrement délicate à implémenter correctement. L'algorithme hongrois requiert:
\begin{itemize}
    \item Une compréhension fine des conditions de KKT (Karush–Kuhn–Tucker)
    \item Une gestion complexe des potentiels duaux
    \item Une optimisation constante pour maintenir la complexité théorique
\end{itemize}

\section{Conclusion}
\label{sec:conclusion}

Pour conclure, ce projet nous a permis d'approfondir notre compréhension des problèmes de correspondance dans les graphes, en particulier dans le contexte de la recherche de solutions optimales via des algorithmes bien établis tels que l'algorithme Hongrois et l'algorithme de Ford-Fulkerson.

L'algorithme Hongrois, avec son approche d'optimisation linéaire, nous a montré comment résoudre efficacement des problèmes de correspondance à coût minimal dans des graphes bipartites, en garantissant une solution optimale dans des temps raisonnables. D'autre part, l'algorithme de Ford-Fulkerson, basé sur le principe de flux maximal, nous a offert une perspective différente en modélisant le problème comme un réseau de transport, ce qui nous a permis d'explorer une approche fondée sur les graphes de flux pour trouver des correspondances maximales.

Ces algorithmes, bien que théoriquement solides, ont nécessité une mise en œuvre rigoureuse et un travail approfondi pour les adapter à notre problème spécifique. Nous avons dû gérer de multiples contraintes, comme les couleurs et les valeurs des cellules, tout en optimisant le processus de sélection des paires. Le projet a ainsi renforcé notre capacité à appliquer des concepts théoriques à des problèmes pratiques, tout en nous permettant de mieux comprendre les nuances et les défis de l'optimisation combinatoire.

En outre, ce projet a été une véritable expérience de travail en équipe. Chaque membre a apporté ses compétences, tant en programmation qu'en analyse algorithmique, pour faire avancer le projet de manière cohérente et efficace. La collaboration étroite et la rigueur dans le développement et les tests des algorithmes ont été essentielles pour surmonter les défis techniques et garantir la réussite de l'implémentation. Ce travail collectif a enrichi notre approche du problème et nous a permis de tirer des leçons importantes sur la gestion de projets complexes et la collaboration dans le développement logiciel.

\section*{Remerciements}

Nous tenions à remercier M. Benomar qui nous a accompagné au fil de la réalisation de ce projet. Sans lui, nous aurions eu beaucoup de mal à nous approprier les notions de théorie des graphes nécessaires à la résolution dudit problème.

Nous tenions aussi à remercier M. Galiana pour les explications claires quant à la mise en place d'un environnement python fonctionnel.

\end{document}